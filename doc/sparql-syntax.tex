\documentclass[12pt]{article}

\usepackage[utf8]{inputenc}
\usepackage[spanish]{babel} 
\usepackage{listings}

\title{Soporte al nivel del lenguaje de programación para el desarrollo de aplicaciones en la web semántica}
\author{
        Mariano Montone \\
        marianomontone@gmail.com \\
        mmontone@github.com
}
\date{\today}

\begin{document}
\maketitle

\begin{abstract}
En este informe presentamos algunas extensiones al nivel del lenguaje de programacion para el desarrollo de aplicaciones en la web semántica.
\end{abstract}

\section{Introducci\'on}
\subsection{Lenguajes de programaci\'on y extensibilidad}

\subsection{La web sem\'antica}

Elementos de la web sem\'antica:
\begin{itemize}
\item URIs
\item Literales
\item SPARQL  
\end{itemize}

\section{Enfoque tradicional}

Para URIs y manejo de literales, existen objetos apropiados. Por ejemplo, desarrollando en Java Jena, tanto URIs y literales son objetos de primer orden.

Para la escritura de consultas SPARQL, el desarrollador puede optar por concatenar cadenas de caracteres. Por ejemplo:

\begin{lstlisting}
String queryString = "PREFIX rdfs: <http://www.w3.org/2000/01/rdf-schema#>"
		+ "PREFIX rdf: <http://www.w3.org/1999/02/22-rdf-syntax-ns#>"
		+ "PREFIX purl: <http://purl.org/dc/terms/>"
		+ "PREFIX twss: <http://www.info.unlp.edu.ar/twss/>"
		+ "PREFIX twssprop: <http://www.info.unlp.edu.ar/twss#>"
		+ "SELECT ?p ?v WHERE { <" + loadUri + "> ?p ?v" + "}";}
\end{lstlisting}	

De todas formas, existen otras opciones, como usar ARQ.

Otra alternativa, seria diseñar un DSL embebido. Algo como:

\begin{lstlisting}
Query query = ResourceCreator.createQuery().
                     prefix(RDF.rdf).
                     prefix(TWSS.twss).
                     select("?p ?v").
                     where({loadUri, "?p", "?v"});
\end{lstlisting}		

\section{Extensiones sint\'acticas}

Ser\'ia interesante si fuese imposible integrar a nivel sint\'actico, especialmente la construcci\'on de consultas.

\begin{lstlisting}
Query query = SPARQL PREFIX rdfs: <http://www.w3.org/2000/01/rdf-schema#>
		             PREFIX rdf: <http://www.w3.org/1999/02/22-rdf-syntax-ns#>
		             PREFIX purl: <http://purl.org/dc/terms/>
		             PREFIX twss: <http://www.info.unlp.edu.ar/twss/>
		             PREFIX twssprop: <http://www.info.unlp.edu.ar/twss#>
		             SELECT ?p ?v WHERE { loadUri ?p ?v }
		      END SPARQL;
\end{lstlisting}

Esto es posible de hacer, pero es necesario implementar un preprocesador de c\'odigo que haga el parsing de la consulta.

Resulta que los lenguajes de la familia Lisp son especialmente aptos para esto, por contener un preprocesador ''builtin'' en la forma de macros. 

Mediante dos cosas: macros, que permiten hacer transformaciones de c\'odigo en tiempo de compilaci\'on; y la extensi\'on del lexer del ambiente. Utilizaremos el primero para implementar la integraci\'on de consultas SPARQL, y el segundo para el manejo literal de URIs. 

\subsection{URIs literales}

Por un lado, definimos los prefijos que bamos a utilizar en las URIs:

\begin{lstlisting}
(define-uri-prefix twss "http://www.info.unlp.edu.ar/twss")
\end{lstlisting}

Luego, podemos utilizarlos en nuestras URIs. La sintaxis literal para URIs tiene la siguiente forma:

\begin{lstlisting}
#<uri>
\end{lstlisting}

o

\begin{lstlisting}
#<prefix:uri>
\end{lstlisting}

Por ejemplo:

\begin{lstlisting}
#<http://www.info.unlp.edu.ar/twss/Person/22> 

=> 

#<PURI:URI http://www.info.unlp.edu.ar/twss/Person/22>
\end{lstlisting}

que es equivalente a:

\begin{lstlisting}
#<twss:/Person/22> 

=> 

#<PURI:URI http://www.info.unlp.edu.ar/twss/Person/22>
\end{lstlisting}

En los ejemplos anteriores,  \verb=#<PURI:URI http://www.info.unlp.edu.ar/twss/Person/22>= es el objeto resultante de aplicar la sintaxis literal, un objeto de tipo URI. Y el s\'imbolo \verb#=># indica el resultado de la evaluaci\'on.

\subsection{Sintaxis para consultas SPARQL}

A trav\'es de la macro \verb=sparql=:
\begin{lstlisting}
(sparql (:select ?x :where (?x ?y ?z)))
\end{lstlisting}

Es interesante que el lenguaje est\'a embebido correctamente, es posible acceder a las variables definidas afuera de forma esperada. Por ejemplo:

\begin{lstlisting}
(let ((value "hello"))
	 (sparql (:select * :where (?x ?y value))))
	 
=> 

"SELECT * WHERE { ?x ?y \"hello\"}"	 
\end{lstlisting}

Por \'ultimo, resulta \'util la combinaci\'on de URIs literales en las consultas SPARQL:

\begin{lstlisting}
(sparql (:select * :where (?x ?y ?z)
      		(?x #<rdf:type> "MyType")))

=>			

"SELECT * WHERE { ?X ?Y ?Z . ?X <http://www.w3.org/1999/02/22-rdf-syntax-ns#type> \"MyType\"}"
\end{lstlisting}

\section{Conclusiones}\label{conclusions}

Si bien puede que estas extensiones no sean criticas y en verdad lenguajes como Java y el framework Jena hacen un buen manejo de recursos en aplicaciones en la web semantica, este trabajo sirve para explorar otras posibilidades.

\bibliographystyle{abbrv}
\bibliography{main}

\end{document}